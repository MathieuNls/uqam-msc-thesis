\begin{introduction}

\section*{Contexte de l'\'etude: Les applications \`a base de services et les anti-patrons}

Les syst\`emes \`a base de service (SBSs) sont compos\'es de services d\'ej\`a pr\^ets et accessibles par Internet \citep{Erl2008}.
Les services sont des unit\'es logicielles autonomes, interop\'erables, et r\'eutilisables qui peuvent \^etre impl\'ement\'ees en utilisant un large choix de technologies tels que les services web, REST (\textit{REpresentational State Transfert}), ou SCA (\textit{Service Component Architecture}, une surcouche au SOA).
La plupart des plateformes web connues du grand public sont de parfaits exemples de SBSs comme Amazon, PayPal et eBay.
De tels syt\`emes sont complexes --- ils g\'en\`erent des fl\^ots massifs de communication entre les services --- et hautement dynamiques: des services apparaissent, disparaissent ou sont modifi\'es.
L'\'evolution constante des SOAs (\textit{Service Oriented Architecture Systems}) peut facilement d\'et\'eriorer la qualit\'e de leurs architectures par l'introduction de d\'efauts architecturaux connus sous le nom d'anti-patrons SOA \citep{Moha}.
Un anti-patron de conception est une solution connue et non-optimale \`a un probl\`eme connu.
Les anti-patrons sont oppos\'es aux patrons de conception qui sont eux des bonnes pratiques en r\'eponse \`a des probl\`emes connus.
Les anti-patrons peuvent donc \^etre qualifi\'es de mauvaises pratiques de conception.

Par exemple, le \textit{Tiny Service} est un anti-patron  largement r\'epandu dans les syst\`emes \`a base de services.
Ce service a une tr\`es petite taille avec tr\`es peu de m\'ethodes qui impl\'ementent seulement une partie d'une abstraction \citep{Dudney2003}.
Les \textit{Tiny Services} sont g\'en\'eralement accompagn\'es de plusieurs autres services fortement coupl\'es, ce qui induit une complexit\'e dans le d\'eveloppement et r\'eduit la r\'eutilisabilit\'e.
De plus, il a \'et\'e prouv\'e que les \textit{Tiny Services} sont une des principales raisons d'\'echecs (\textit{failure}) des syst\`emes \`a base de services \citep{Kral2009}.


\section*{Probl\`eme \'etudi\'e : La d\'etection des anti-patrons}

\'Etant donn\'e l'impact n\'efaste des anti-patrons sur la r\'eutilisabilit\'e et la maintenabilit\'e des SBS, il existe un besoin clair et urgent de techniques et d'outils visant leur d\'etection.
Cependant, la nature hautement dynamique et distribu\'ee des SBSs rend la d\'etection automatique des anti-patrons SOA compliqu\'ee.
C'est un v\'eritable d\'efi, surtout en comparaison avec d'autres outils visant la d\'etection d'anti-patrons dans les syst\`emes objets \citep{Marinescu2004, Fokaefs2007, Moha2010}.
En 2012, notre \'equipe, compos\'ee de Naouel Moha, Francis Palma, Benjamin Joyen-Conseil, Yann-Ga\"el Gu\'eh\'eneuc, Benoit Beaudry, Jean-Marc J\'ez\'equel et moi m\^eme, a d\'evelopp\'e une approche nomm\'ee SODA (\textit{Service Oriented Detection for Anti-patterns}) \citep{Nayrolles, Moha} qui vise la d\'etection des anti-patrons SOA.
Cette approche repose sur un langage sp\'ecifique au domaine (\textit{Domain Specific Language}, DSL) pour sp\'ecifier les anti-patrons SOA et est bas\'ee sur des m\'etriques (majoritairement statiques) d'un c\^ot\'e, et sur une m\'ethode de g\'en\'eration automatique d'algorithmes de d\'etection, de l'autre.


Bien qu'\'etant efficace et pr\'ecise, SODA souffre de s\'erieuses limitations.
En effet, SODA ex\'ecute deux phases d'analyse, une premi\`ere statique, suivie d'une seconde, dynamique.
La premi\`ere analyse statique requiert un acc\`es aux interfaces des services.
En cons\'equence, SODA ne peut pas analyser dont les sources ne sont pas disponible.
La seconde phase, \red{\`a moindre port\'ee}, d'analyse dynamique requiert l'\'ex\'ecution concr\`ete du syst\`eme et de ce fait, la cr\'eation de sc\'enarios ex\'ecutables.
De plus, SODA a \'et\'e cr\'e\'e sp\'ecifiquement pour les syst\`emes de type SCA et sa pr\'ecision tend \`a faiblir lorsque la taille des syst\`emes augmente.
\'Etant donn\'e les limitations de SODA, il y a un \red{espace pour l'am\'elioration de} nos approches et outils pour la d\'etection d'anti-patrons SOA.
Ces am\'eliorations doivent apporter une d\'etection pr\'ecise, efficace et applicable \`a toutes les technologies SOA (REST, SCA, service web, ...) et ce quelque soit la taille du syst\`eme.
Dans ce m\'emoire, nous proposons une approche nomm\'ee SOMAD (\textit{Service Oriented Mining for AntiPatterns Detection}).
Cette approche ne requiert pas de sc\'enarios, \`a l'inverse de SODA, et repose uniquement sur les traces d'ex\'ecution qui peuvent \^etre disponibles dans toutes les technologies SOA.
SOMAD est capable d'\'eliminer les donn\'ees non pertinentes en utilisant une technique de fouille de donn\'ees: la fouille de r\`egles d'association s\'equentielle.
La fouille de r\`egles d'association \citep{Agrawal1994} et a fortiori la fouille de r\`egles d'association s\'equentielles -- en particulier l'algorithme \textit{RuleGrowth} \citep{Fournier-viger2011} -- sont des m\'ethodes servant \`a d\'ecouvrir des relations int\'eressantes dans de larges bases de donn\'ees.
SOMAD \red{applique} ses m\'ethodes sur les traces d'ex\'ecution pour d\'ecouvrir des anti-patrons SOA \red{sous forme de configurations particuli\`eres dans la composition des r\`egles d'association.} Pour ce faire, nous utilisons une variante de la fouille de r\`egles d'association bas\'ee sur les s\'equences ou \'episodes.
Dans notre cas, les s\'equences repr\'esentent \red{des suites d'appels} de services et de m\'ethodes.
Ensuite, nous filtrons ces r\`egles d'association en utilisant une suite de m\'etriques \red{d\'edi\'ees afin d'}extraire la connaissance pertinente des r\`egles d'association.

\section*{Contributions}

Les principales contributions li\'ees \`a ce m\'emoire sont les suivantes:

\begin{enumerate}
\item Une nouvelle approche nomm\'ee SOMAD pour la d\'etection des anti-patrons SOA.
Cette approche est bas\'ee sur des r\`egles d'association fouill\'ees sur des traces d'ex\'ecution qui peuvent provenir de toutes les technologies SOA.
\item une validation empirique de notre approche qui d\'emontre l'am\'elioration apport\'ee par SOMAD en terme de pr\'ecision (8.3\% \`a 20\%) et de vitesse (2.5 fois plus rapide).
\item Une \'evolution de l'algorithme \textit{RuleGrowth} visant la fouille de r\`egles d'association s\'equentielles dans des s\'equences d'appels des SBSs.
De ce fait, nous am\'eliorons la d\'etection d'anti-patrons SOA.
\end{enumerate}

Les contributions 1 et 2 ont \'et\'e pr\'esent\'ees \`a la $20^{ieme}$ \'edition de la conf\'erence internationale de travail sur la r\'etro-ing\'enierie (WCRE 2013, \textit{Working Conference on Reverse Engineering}) \citep{Nayrolles2013a}.

\section*{Structure du document}

Ce m\'emoire est organis\'e de la mani\`ere suivante. Les deux premiers chapitres pr\'esentent respectivement l'\'etat de l'art sur la d\'etection de patrons de conceptions et l'extraction de connaissances depuis les traces d'ex\'ecution.
Le troisi\'eme chapitre, quant \`a lui, pr\'esente l'approche SOMAD.
Le quatri\'eme pr\'esente l'impl\'ementation de \textsc{Somad} tandis que le cinqui\`eme et dernier chapitre pr\'esente nos exp\'erimentations. Enfin, nous proposons quelques remarques de fin dans la conclusion.

\end{introduction}